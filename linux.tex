 \documentclass{article}
\usepackage{listings}
\usepackage{xcolor}
\usepackage{hyperref}

\title{\textbf{Linux Hardening Guide}}
\author{Kevin Schoonver, Gavin Lewis}

\lstset{basicstyle=\ttfamily,
    showstringspaces=false,
    commentstyle=\color{red},
    keywordstyle=\color{blue},
    breaklines=true,
}

\begin{document}
\maketitle

\tableofcontents 

\pagebreak 

\section{Setup Up Package Repository}
The majority of the linux boxes you will be faced with will not have the proper
package repositories to collect packages. Whether it be a Debian, Ubuntu, or
Fedora, if the version is no longer support by the maintainers of the distro,
you will need to change the location in which the package manager will look for
repositories. The distro probably supports some sort of archive in which the
packages are kept, but no longer updated. 

Please view the sections below to update the package repositories on various
systems.

\subsection{Ubuntu}
\begin{lstlisting}[language=bash]
#!/bin/bash
cp /etc/apt/sources.list /etc/apt/sources.list.bak # Backup sources list
sudo sed -i -re "s/([a-z]{2}\.)?archive.ubuntu.com|security.ubuntu.com/old-releases.ubuntu.com/g" /etc/apt/sources.list
\end{lstlisting}

\subsection{Debian}
\begin{lstlisting}[language=bash]
#!/bin/bash
cp /etc/apt/sources.list /etc/apt/sources.list.bak # Backup sources list
sed "s/^deb/#deb/g" -i /etc/apt/sources.list  # Comment old sources

cat << EOF >> /etc/apt/sources.list
deb http://archive.debian.org/debian/ <version> main non-free contrib
deb-src http://archive.debian.org/debian/ <version> main non-free contrib

deb http://archive.debian.org/debian-security/ <version>/updates main non-free contrib
deb-src http://archive.debian.org/debian-security/ <version>/updates main non-free contrib
EOF
\end{lstlisting}

\subsection{CentOS}
\begin{lstlisting}[language=bash]
#!/bin/bash
cp /etc/yum.repos.d/CentOS-Base.repo /etc/yum.repos.d/CentOS-Base.repo.bak # Backup sources
cp /etc/yum.repos.d/CentOS-Vault.repo /etc/yum.repos.d/CentOS-Base.repo # Overwrite with Vault
sed -i -re "s/enabled.=.1/enabled = 0/g" /etc/yum.repos.d/rpmforge.repo
sed -i -re "s/enabled.=.0/enabled = 1/g" /etc/yum.repos.d/CentOS-Base.repo
\end{lstlisting}


\section{User Management}
\subsection{Check .bashrc}
Check the local \textbf{.bashrc} to ensure that any rogue commands are not run.
If there is anything malicious in the \textbf{.bashrc}, remove and then reboot
the machine.

\subsection{Change Administrator Passwords}
Utilize the password format excel document to change the account of any
high-priviledge accounts on the machine. As well as important system users such
as the mysql account.

\begin{lstlisting}[language=bash]
#!/bin/bash
passwd <user> # Exclude <user> for changing current account
\end{lstlisting}

\subsection{Create New Administrator Account}
Make a new administrator account which will be used for all future
administrative tasks. Disable any other administrative accounts which can be
used by the attackers to run commands as an elevated user.

\begin{lstlisting}[language=bash]
#!/bin/bash
adduser # Creates a new user
usermod -a -G sudo USERNAME # Add user to sudo
su USERNAME # Change to the new uesr
sudo ls # Test that the sudo works
\end{lstlisting}

\subsection{Check for suspicious accounts}
Check the system groups to determine if certain suspicious accounts are in
groups they should not be. To check all the groups, run the following command:
\begin{lstlisting}[language=bash]
#!/bin/bash
cat /etc/group | sort | less
\end{lstlisting}

Be sure to check the normal administrative groups such as \textbf{sudo},
\textbf{wheel}, \textbf{admin}, and \textbf{root}. Make note of these suspicious
accounts. 

Next, check all of the users on the system using the following command:
\begin{lstlisting}[language=bash]
#!/bin/bash
less /etc/passwd
\end{lstlisting}
The users in this file are listed in chronological order based on when they were
added so users later in the list are more likely to be rogue accounts. For
service accounts such as \textbf{avahi}, they SHOULD NOT have a login shell nor
a home directory such as \\
\textbf{avahi}:x:107:114:Avahi mDNS daemon,,,:\textbf{/home/bob:/bin/bash} \\
Daemons like Avahi should have their shells set to \textbf{/bin/false},
\textbf{/sbin/nologin}, or \textbf{/usr/sbin/nologin}. 

Lastly, check the \textbf{sudoers} file. This should only contain administrative
accounts such as \textbf{root} or \textbf{admin}. Use the command:
\begin{lstlisting}[language=bash]
#!/bin/bash
less /etc/sudoers
\end{lstlisting}

\subsection{Disable / Update Administrator and System Accounts}
Now that you have the new administrator account, you will no longer need any of
the other administrative accounts such as \textbf{root} or
\textbf{administrator}. To disable these accounts, use the following command:
\begin{lstlisting}[language=bash]
#!/bin/bash
usermod --lock --shell /usr/sbin/nologin --expiredate 1970-02-02 USERNAME
\end{lstlisting}

Look for system / services accounts which are used to run various services or
applications. 

\subsection{Check Sudoers Group}
There should only be a couple of accounts in this group, the one you created and
the \textbf{root} or \textbf{administrator} account that came with the box.


\section{Check for suspicious files}
Follows are methods used to find suspicious files and check valid files for changes.

\subsection{/tmp/}
Since the /tmp/ folder is utilized primiarly to store temporary data, malware or
other malicious programs can mount sockets or hide other files in the /tmp/
folder.  Use the following command to check for these files.

\begin{lstlisting}[language=bash]
#!/bin/bash
ls -al /tmp/
\end{lstlisting}

If anything in this folder looks suspicious (or you just want to be safe), create
an incident report about the file and run the following command.

\begin{lstlisting}[language=bash]
#!/bin/bash
rm -rf /tmp/<file> # rm -rf /tmp/* to delete everything
\end{lstlisting}

\subsection{Root Kit Hunter}
\textbf{rkhunter} is a tool used to find root kits on a machine.

\begin{lstlisting}[language=bash]
#!/bin/bash
sudo apt-get install rkhunter
sudo rkhunter --version
\end{lstlisting}

Use this or the variant for your OS. The second line will make sure it is up to
date. The current latest version is 1.4.0.

\begin{lstlisting}[language=bash]
#!/bin/bash
sudo rkhunter --update
\end{lstlisting}

This line will update the check files for \textbf{rkhunter}.

\subsection{Installing Anti-Virus}
Installing an Anti-Virus on your box will help immensely. They also typically end
up as an inject, so installing and running an anti-virus will be helpful no matter
what. The one installed will depend on the OS.

\subsubsection{Linux}
\textbf{clamav} will be used to find infected files on the system.

\begin{lstlisting}[language=bash]
#!/bin/bash
sudo apt-get install clamav
sudo freshclam
\end{lstlisting}

The pair of commands will install the latest version of \textbf{clamav} as well as
update all virus definitions.

\begin{lstlisting}[language=bash]
#!/bin/bash
clamscan -r --move=/somewhere / &
\end{lstlisting}

This command will scan the entire machine, moving any malicious files to a folder
called \textbf{/somewhere}. The \textbf{\&} is there to run the command in the 
background, but this can be accomplished with \textbf{tmux} or any other known way.

\subsubsection{Windows}
Sophos


\section{Check cron jobs}
Cron jobs are scheduled "jobs" or commands fo rthe server to run automatically at
a specified time and date. Malware or the redteam may attempt to use these for
persistence mechanisms so clearing them is very important.

To check the cron jobs of a specific user, use:

\begin{lstlisting}[language=bash]
#!/bin/bash
crontab -l <user>
\end{lstlisting}

To clear any suspicious cron jobs, use:
\begin{lstlisting}[language=bash]
#!/bin/bash
crontab -r <user>
\end{lstlisting}

\section{Checking for Suspicious Programs}
Some of the boxes may already be infected and have malicious processes running.
To view a list of processes use:

\begin{lstlisting}[language=bash]
#!/bin/bash
ps -eo pid,tty,user,args | less # List all process with the following format
\end{lstlisting}

As a rule of thumb, programs should only be running in \textbf{/bin},
\textbf{/opt}, \textbf{/sbin}, \textbf{/usr}.

\section{Setup Splunk}

\section{Setup NTP}
NTP stands for Network Time Protocol and runs on port 123. It is used to synchronize computers on a network to UTC time within a few miliseconds. This should not be necessary during CCDC unless there is an inject requiring it, or there is extra time. Keeping similar times on each 
machine will help with logging and is good practice to have done.

\subsection{Creating NTP Server}
Choose one of the Linux machines to operate as the NTP server. This machine will be the one
acting as the lowest stratum to the rest of the computers. Run the following commands.

\begin{lstlisting}[language=bash]
#!/bin/bash
sudo apt-get install ntp
\end{lstlisting}

Use the package manager that is on the machine. Next edit \textbf{/etc/ntp.conf}. Make sure the following lines are present. They restrict access to clients. Replace the addresses in {} with the actual subnet of the network given to us. Make sure to delete the brackets.

\begin{lstlisting}[language=bash]
#!/bin/bash
restrict default kod nomodify notrap nopeer noquery
restrict -6 default kod nomodify notrap nopeer noquery
restrict {OUR.NET.WOR.K} mask {255.255.255.0} nomodify notrap
\end{lstlisting}

The next lines will use the systems clock in case Internet access is lost.

\begin{lstlisting}[language=bash]
#!/bin/bash
server  127.127.1.0 # local clock
fudge   127.127.1.0 stratum 10
\end{lstlisting}

Start the service. The NTP server is ready for clients to be configured.

\begin{lstlisting}[language=bash]
#!/bin/bash
sudo service ntp restart
\end{lstlisting}


\subsection{Configuring Clients}
The rest of the computers must be configured to use the correct NTP server.

\subsubsection{Windows}


\subsubsection{Linux}
All other machines will act as NTP clients. Use the correct package manager for your system.

\begin{lstlisting}[language=bash]
#!/bin/bash
sudo apt-get install ntp
\end{lstlisting}

Edit the \textbf{/etc/ntp.conf} file to add the NTP server to the machine. This line will be added near the top of the file above where the other servers are declared.

\begin{lstlisting}[language=bash]
# Only add the line directly below me
server NTP.SER.VER.IP prefer
server 0.debian.whatever iburst # this is already there
\end{lstlisting}

The next two commands will restart the NTP service, and then list all NTP servers the client sees.

\begin{lstlisting}[language=bash]
#!/bin/bash
sudo service ntp restart
ntpq -p
\end{lstlisting}


\section{Looking at Logs}
Look at em

\section{Using Mitnik}
Mitnik is the Discord bot that helps out during the competition. He can be used to generate passwords and assist in generating Incident reports. Use \textbf{?help} to access his help menu.

\subsection{?genpass}
This command will return a new password consisting of a symbol, three words, followed by a number between 00 and 99. 

\subsection{?incident}
This is the command to use to create, edit and list different incidents.

\subsubsection{Create}
Create an incident from scratch.

\begin{lstlisting}[language=bash]
?incident -n
\end{lstlisting}

This will begin an interface with the bot to create a new incident. Follow the prompts on screen while being as descriptive as possible. Everything typed goes straight into the official report, so make it detailed enough to make sense to someone somewhat familiar with the network. 

\subsubsection{List}
List all or a single incident.

\begin{lstlisting}[language=bash]
?incident -l {NUM}
\end{lstlisting}

Leaving out a specific incident number will list all incidents and their included data. This should be used before editing incidents so the incident number can be found. When run with a number, the command will return only the specified incident.

\subsubsection{Edit}
Edit an already created incident.

\begin{lstlisting}[language=bash]
?incident -e {NUM}
\end{lstlisting}

Will edit the specified incident. To find an incidents number, use the list command. Follow the prompts from Mitnik to edit attributes of the incident.s


\section{Sources}
\begin{enumerate}
    \item Initial: \url{http://www.cs.mercer.edu/courses/David%20Cozart/IST%20301/Cyber%20Defense%20Spring%202017/Hardening%20Directives/LinuxGeneralDocumentation-Adams.pdf} 
   	\item NTP:
   	\url{https://www.thegeekstuff.com/2014/06/linux-ntp-server-client/}
\end{enumerate}


\end{document}
